\documentclass[12pt,a4paper,oneside]{article}
\usepackage{fancyhdr}
\usepackage{geometry}
\usepackage{graphicx}
\usepackage{lastpage}
\usepackage{wrapfig}
\usepackage{tikz}
\usepackage{eso-pic}
\usetikzlibrary{calc}
\usetikzlibrary{decorations.pathmorphing}
\usepackage[utf8]{inputenc} % Required for inputting international characters
\usepackage[T1]{fontenc} % Output font encoding for international characters
\usepackage{mathpazo} % Palatino font


\AddToShipoutPictureBG{
  \begin{tikzpicture}[overlay,remember picture]
    \draw [line width=3pt]
      ($ (current page.north west) + (1.5cm,-2.0cm) $)
      rectangle
      ($ (current page.south east) + (-1.5cm,1.8cm) $);
    \draw [line width=1pt]
      ($ (current page.north west) + (1.65cm,-2.15cm) $)
      rectangle
      ($ (current page.south east) + (-1.65cm,1.95cm) $); 
    {\ifnum\value{page}>1 
        \draw [line width=2pt]
          ($ (current page.north west) + (1.65cm,-4.35cm) $)
          --
          ($ (current page.north east) + (-1.65cm,-4.35cm) $); 
        \draw [line width=2pt]
          ($ (current page.north west) + (1.65cm,-4.55cm) $)
          --
          ($ (current page.north east) + (-1.65cm,-4.55cm) $);

        \draw [line width=2pt]
          ($ (current page.center) + (0.25cm,12.7cm) $)
          --
          ($ (current page.center) + (0.25cm,10.5cm) $);
        \draw [line width=2pt]
          ($ (current page.center) + (0.05cm,12.7cm) $)
          --
          ($ (current page.center) + (0.05cm,10.5cm) $);

        \draw [line width=2pt]
          ($ (current page.south west) + (1.65cm,2.8cm) $)
          --
          ($ (current page.south east) + (-1.65cm,2.8cm) $); 
        \draw [line width=2pt]
          ($ (current page.south west) + (1.65cm,3cm) $)
          --
          ($ (current page.south east) + (-1.65cm,3cm) $);

        \draw [line width=2pt]
          ($ (current page.center) + (0.25cm,-12.9cm) $)
          --
          ($ (current page.center) + (0.25cm,-12cm) $);
        \draw [line width=2pt]
          ($ (current page.center) + (0.05cm,-12.9cm) $)
          --
          ($ (current page.center) + (0.05cm,-12cm) $);
    \fi}
  \end{tikzpicture}%
}

\makeatletter
\newcommand{\subtitledoc}[1]{\newcommand{\@subtitledoc}{#1}}
\newcommand{\instituto}[1]{\newcommand{\@instituto}{#1}}
\newcommand{\carrera}[1]{\newcommand{\@carrera}{#1}}
\newcommand{\professorTeo}[1]{\newcommand{\@professorTeo}{#1}}
\newcommand{\professorPrac}[1]{\newcommand{\@professorPrac}{#1}}
\newcommand{\catedra}[1]{\newcommand{\@catedra}{#1}}
\newcommand{\curso}[1]{\newcommand{\@curso}{#1}}
\newcommand{\legajo}[1]{\newcommand{\@legajo}{#1}}
\newcommand{\footerauthor}[1]{\newcommand{\@footerauthor}{#1}}
\newcommand{\footerlegajo}[1]{\newcommand{\@footerlegajo}{#1}}
\newcommand{\footercatedra}[1]{\newcommand{\@footercatedra}{#1}}


\geometry{a4paper,left=1in,right=1in}
\setlength\headheight{59.58496pt}
\addtolength{\topmargin}{-0.00496pt}
\setlength{\footskip}{42pt}
\fancyheadoffset[L]{7.5mm}
\fancyheadoffset[R]{7.5mm}
\fancyfootoffset[L]{7.5mm}
\fancyfootoffset[R]{7.5mm}

\fancyhead[C]{} %eliminar antiguo numero de pagina
\fancyhead[L]{
  \begin{minipage}[b]{66mm}
    \footnotesize{
      \textbf{UTN} - Universidad Tecnológica Nacional \\
      \textbf{Especialidad: }Electrónica\\
      \textbf{Cátedra: } \@catedra\\
      \textbf{Comisión: }\@curso\\
      \textbf{Página: } \thepage\ de \pageref{LastPage}
    }
  \end{minipage}
  \begin{minipage}[b]{7.7mm}
    \includegraphics[width=15mm]{./inc/COPYlogo-utn.png}
  \end{minipage}
}
\fancyhead[R]{
  \begin{minipage}[b]{80mm}
    \raggedright
    \textbf{Profesor de teórico:} \small{\@professorTeo} \\
    \textbf{Profesor de practico:} \@professorPrac \\
    \textbf{Alum.:} \@footerauthor \\
    \textbf{\@title}
  \end{minipage}
}

\fancyfoot[C]{} %eliminar antiguo numero de pagina
\fancyfoot[R]{Página \thepage\ de \pageref{LastPage}}
\fancyfoot[L]{\@date}

\renewcommand{\headrulewidth}{0pt}
\pagestyle{fancy}

\renewcommand{\maketitle}{
  \newpage
  \thispagestyle{empty}
  \begin{center}
    \includegraphics[width=0.32\textwidth]{inc/COPYlogo-utn.png}\\
    \huge{
      UTN
      \medskip
      ``Universidad Tecnológica Nacional''
    }
    \vspace{0.5cm}
  \end{center}
  \par
  {\Large \textbf{Catedra:} \@catedra\par
    \textbf{Profesor de teórico:} \@professorTeo\par
    \textbf{Profesor de practico:} \@professorPrac\par
    \textbf{Integrantes:}\@author\par
  }
  \textbf{\Large{\@title}}
  \vspace{1cm}
  \begin{center}
    {\Large\@date}
  \end{center}
  \newpage
}
\makeatother

\usepackage[spanish]{babel}
\usepackage{amsmath}
\usepackage{booktabs}
\usepackage{xfrac}
\usepackage{siunitx}
\AddToHook{cmd/section/before}{\clearpage}
\newcommand{\parallelsum}{\mathbin{\!/\mkern-5mu/\!}}
\usepackage{array}
\newcolumntype{P}[1]{>{\centering\arraybackslash}p{#1}}

\catedra{Dispositivos Electrónicos}
\title{TP1: ``Leyes de Ohm y de Kirchhoff''}
\footerauthor{Avedano, Giorgis, Gomez, Petiti}
\curso{3R4}
\professorTeo{Avramovich Alejandro}
\professorPrac{}
\author{
  \begin{itemize}
    \item Avedano Nicolas 
    \item Giorgis Ezequiel
    \item Gomez Enzo
    \item Petiti Matias
  \end{itemize}
}

\begin{document}
\maketitle
\tableofcontents
\newpage
\section{Introducción}
En este trabajo práctico de laboratorio implementaremos las leyes de Ohm y Kirchhoff  para calcular circuitos de resistores. De cada diagrama calcularemos la resistencia total, las caídas de tensión y las corrientes. Luego, armaremos los circuitos en la protoboard y por medio de mediciones, demostraremos de manera experimental  que las leyes de ohm y Kirchhoff se cumplen. Para finalizar, compararemos los datos calculados con los medidos y calcularemos el porcentaje de error
\section{Marco Teórico}
\subsection{Ley de Ohm}
Hay una relación fundamental entre las tres magnitudes básicas de todos los circuitos eléctricos/electrónicos.
\begin{align}
  I=\frac{V}{R} 
\end{align}
La intensidad (corriente) que recorre un circuito es directamente proporcional a la tensión de la fuente de alimentación e inversamente proporcional a la resistencia en dicho circuito.
\subsection{Ley de Kirchhoff de las tensiones}
Esta ley establece: “En un lazo cerrado, la suma de todas las caídas de tensión es igual a la tensión total suministrada. De forma equivalente, la suma algebraica de las diferencias de potencial eléctrico en un lazo es igual a cero.”
\begin{align}
  \sum_{k=1}^{n} V_{R} = V_{1} + V_{2} +V_{3} +\dots+V_{n} = 0
\end{align}
\subsection{Ley de Kirchhoff de las corrientes}
Esta ley también es llamada ley de nodos o primera ley de Kirchhoff. Esta establece: ``En cualquier nodo, la suma de las corrientes que entran en ese nodo es igual a la suma de las corrientes que salen. De forma equivalente, la suma de todas las corrientes que pasan por el nodo es igual a cero''
\begin{align}
  \sum_{k=1}^{n} I_{R} = I_{1} + I_{2} +I_{3} +\dots+I_{n} = 0
\end{align}
\section{Materiales}
\subsection{Componentes}
\begin{center}
\begin{tabular}{c c c}
 Cantidad &Componente &Descripción \\
 \hline
 1&Resistor &$100\unit{\ohm} \;  \sfrac{1}{4}\unit{\watt}$ \\
 1&Resistor &$200\unit{\ohm} \;  \sfrac{1}{4}\unit{\watt}$ \\
 1&Resistor &$150\unit{\ohm} \;  \sfrac{1}{4}\unit{\watt}$ \\
 1&Resistor &$270\unit{\ohm} \;  \sfrac{1}{4}\unit{\watt}$ \\
 2&Resistores &$68\unit{\ohm}\;  \sfrac{1}{4}\unit{\watt}$ 
\end{tabular}
\end{center}
\subsection{Herramientas}
\begin{center}
  \begin{tabular}{c c c}
    Cantidad &Herramienta &Descripción\\ 
    \hline
    1 &Pinza &Alicate \\
    1 &Placa de proyectos \\
    1 &Multímetro
\end{tabular}
\end{center}
\subsection{Instrumentos}
\begin{center}
  \begin{tabular}{c c c}
    Cantidad &Instrumento &Descripción\\ 
    \hline
    1 &Fuente de alimentación &variable
\end{tabular}
\end{center}
\section{Procedimiento}
\begin{itemize}
    \item Medir cada Resistor y Anotar el valor en la tabla. \\
    \item Armar circuito y Medir valor de Resistencia total. \\
    \item Calcular corriente y caídas de tensión de cada Resistor. \\
    \item Conectar Fuente y Medir caídas de tensión y Corrientes. \\
    \item Verificar que se cumplan las leyes de Kirchhoff. 
\end{itemize}
\section{Cálculo, Mediciones, Gráficos, Programa}
\subsection{Circuito 1}
Calculando resistencia total
\begin{align}
  \begin{aligned}
R_{5} &= R_{3} + R_{4} \\
        &= 200\unit{\ohm} + 270\unit{\ohm} \\
        &= 470\unit{\ohm} \\
\end{aligned}
\quad\quad
\begin{aligned}
  R_{5\parallelsum 2}  &= \frac{R_{5}R_{2}}{R_{5}R_{2}} \\ 
                       &= \frac{470\unit{\ohm}150\unit{\ohm}}{470\unit{\ohm}+150\unit{\ohm}} \\ 
                       &=113.70\unit{\ohm}
\end{aligned}
\quad\quad 
\begin{aligned}
  R_{T} &= R_{5\parallelsum 2} + R_{1} \\ 
        &=113.70 \unit{\ohm} + 100 \unit{\ohm} \\
        &= 213.70 \unit{\ohm}
\end{aligned}
\end{align}
Despejando corriente por medio de Ley de Ohm
\begin{align}
  I_{1} = \frac{\varepsilon}{R_{T}} = \frac{10V}{213.70\Omega} = 0.046A = 46.79\unit{\milli\ampere}
\end{align}
Valuado para $R_{1}$
\begin{align}
  V_{R_{1}} = I_{1} R_{1} = 46.79mA \cdot 100\unit{\ohm} = 4.679V 
\end{align}
Por ley de Kirchhoff de las tensiones
\begin{align}
  V_{R_{2}} = \varepsilon - V_{R_{1}} = 10V-4.679V = 5.321V 
\end{align}
Utilizando el valor anterior, por Ley de Ohm
\begin{align}
  I_{2} = \frac{V_{R_{2}}}{R_{2}} = \frac{5.321}{150} = 35.47mA 
\end{align}
Por ley de Kirchhoff de las corrientes
\begin{align}
  I_{3} = I_{1}-I_{2} = 46.79mA - 35.47mA = 11.32mA 
\end{align}
Finalmente
\begin{align}
  V_{R_{3}} = R_{3} I_{3} = 200\unit{\ohm} \cdot 11.32mA  = 2,264V  \\
  V_{R_{4}} = R_{4} I_{3} = 270\unit{\ohm} \cdot 11.32mA  = 3,0564V 
\end{align}

\begin{center}
  \begin{table}
    \begin{tabular}{P{0.17\linewidth} | P{0.17\linewidth} |P{0.17\linewidth} |P{0.17\linewidth} |P{0.17\linewidth}} 
      \toprule
      \textbf{Resistencia (con Óhmetro)} $[\Omega]$ &\textbf{Corriente teórica según Kirchhoff´ [A]} &\textbf{Voltaje teórico según Kirchhoff V= R i [V]} &\textbf{Voltaje en la resistencia, (Medido con Sensor) [V]} &\textbf{Corriente en la resistencia, i’=V’/R [mA]} \\
      R_{1}= &i_{1}=0.0467 &V_{1}=4.6 &V_{1}’=4.62 &i_{1}’=45.5 \\
      R_{2}= &i_{2}=0.0354 &V_{2}=5.3 &V_{2}’=5.27 &i_{2}’=34.7 \\
      R_{3}= &i_{3}=0.0113 &V_{3}=2.26 &V_{3}’=2.22 &i_{3}’=11.1 \\
      R_{4}= &i_{4}=0.0113 &V_{4}=3.056 &V_{4}’=3 &i_{4}’=34.7 \\
      R_{T}= 213.70 &iT= 0.0467 &VT= 10 &V_{T}’=9.91 &i_{T}’=45.5 \\
      \bottomrule
    \end{tabular}
  \end{table} 
\end{center}

\end{document}
